%%%%%%%%%%%%%%%%%%%%%%%%%%%%%%%%%%%%%%%%%%%%%%%%%%%%%%%%%%%%%%%%%%%%%%%%%%%%
% Formato en Látex para la generación de material de estudio en la ESPE - DECE 
% Documento generado en base a los formatos ejemplo otorgados por el Ing. Wilson Cerón
% Por:
% Dillan Aldás
% Andres Jimenez
% 
% Last Update: 2020.04.25 
%%%%%%%%%%%%%%%%%%%%%%%%%%%%%%%%%%%%%%%%%%%%%%%%%%%%%%%%%%%%%%%%%%%%%%%%
\documentclass[a4paper, 11pt]{article}
%Paquetes%%%%%%%%%%%%%%%%%%%%%%%%%%%%%%%%%%%%%%%%%%%%%%%%%%%%%%%%%%%%%%%%%%%
\usepackage[utf8]{inputenc}
\usepackage[T1]{fontenc}
\usepackage[spanish]{babel}
\usepackage{everypage}
\usepackage{graphicx}
\usepackage{lipsum}
\usepackage{tikz}
\usepackage{color}
\usepackage[left=3cm, right=2.75cm, top=4.5cm, bottom=4cm]{geometry}
\usepackage{fancyhdr}
\pagestyle{empty}
\usepackage{tikzpagenodes}
\usetikzlibrary{calc}
\usepackage{hyperref}
\usepackage{lipsum}
\usepackage[user,abspage]{zref}
\usepackage{siunitx}
\usepackage[square,numbers]{natbib}
\usepackage[maxbibnames=99, sorting=none, backend=bibtex]{biblatex}
\addbibresource{referencias.bib}
\usepackage{graphicx} 
\usepackage{spanish}
\usepackage{gensymb}
\usepackage{mathptmx}
\usepackage{amsmath}
\usepackage{calc}
\bibliographystyle{abbrvnat}
%%Comandos y comandos nuevos %%%%%%%%%%%%%%%%%%%%%%%%%%%%%%%%%%%%%%%%%%%%%
%\newcommand{\slink}[2]{\hyperref[#1]{\underline{\smash{#2}}}} 
\renewcommand{\baselinestretch}{1.35} 

%% colores de los hyperlinks - 
% (en Adobe Acrobat los links Sí salen subrayados, en este visor de pdf solo salen azules) 
\hypersetup{%
colorlinks=true, linkcolor=blue, urlcolor=blue,linkbordercolor=blue, citecolor=blue, urlbordercolor=blue, pdfborderstyle={/S/U/W 1}
}
\makeatletter
\Hy@AtBeginDocument{%
  \def\@pdfborder{0 0 1}% Overrides border definition set with colorlinks=true
  \def\@pdfborderstyle{/S/U/W 1}% Overrides border style set with colorlinks=true
                                % Hyperlink border style will be underline of width 1pt
}
\makeatother
%colores%%%%%%%%%%%%%%%%%%%%%%%%%%%%%%%%%%%%%%%%%%%%%%
\definecolor{rojo_oscuro}{rgb}{.85,.45,.45} 
\definecolor{verde_oscuro}{rgb}{0,.5,.2} 
% Formato de todo el documento %%%%%%%%%%%%%%%%%%%%%
\AddEverypageHook{%
\tikz[remember picture,overlay]{%
% Páginas mayores a 2
  \ifnum\value{page}>1%
  % \ifnum\makeatletter\zref@extract{#1}{abspage}\makeatother>1%
  % \ifnum\value{\zref{abspage}}>1%
  % \ifnum\value{page}>1%
    \filldraw[fill=rojo_oscuro, thick,  draw=verde_oscuro] (1,-1.6) rectangle (16,-1.3);
    \filldraw[fill=rojo_oscuro, thick, draw=verde_oscuro] (1,-1.6-22.8) rectangle (16,-1.3-22.8);
        \node[anchor=east] at (16,-25) {\textcolor{black}{{\footnotesize DECE - ESPE}}};
        \node[anchor=west] at (1,-25) {\textcolor{black}{{\footnotesize \thepage}}};
  %logo DECE
        \node at (1.8,0) {\includegraphics[width=2.5cm]{DEEE_LOGO.png}};
%Pagina 1: Carátula%%%%%%%%%%%%%%%%%%%%%%%%%%%%%%%%%%%%%%%%%%%
  \else
    % Márgenes
    \filldraw[fill=rojo_oscuro, thick, draw=verde_oscuro] (-0.3,0) rectangle (0.3,-26); %izquierdo
    \filldraw[fill=rojo_oscuro, thick,  draw=verde_oscuro] (1,-1.6) rectangle (16,-1.3); %arriba
    \filldraw[fill=rojo_oscuro, thick, draw=verde_oscuro] (1,-1.6-21.8) rectangle (16,-1.3-21.8); %abajo
    % Texto de carátula
    \node[anchor=east] at (16,-24) {\textcolor{black}{{\Large Laboratorio de Circuitos eléctricos}}}; 
    \node[anchor=east] at (16,-25) {\textcolor{black}{{\small Procedimiento}}};
    \node[anchor=east] at (16,-25.8) {\textcolor{black}{{\footnotesize 2020}}};
    \node[anchor=east] at (16,-21) {\textcolor{black}{{\Huge Laboratorio 1}}};
    \node[anchor=east] at (16,-22) {\textcolor{black}{{\LARGE Leyes de Kirchhof}}};
    %Logos
    \node at (4.2,0.) {\includegraphics[width=6cm]{ESPE_LOGO.png}};
    \node at (14,0.) {\includegraphics[width=2.5cm]{DEEE_LOGO.png}};
 \fi}
 }
%
%
%
%
% \pagenumbering{roman}
%%%%%%%%%%%%%%%%%%%%%%%%%%%%%%%%%%%%%%%%%%%%%%%%%%%%%%%%%%%%%%%%%%%%%%%%%%%%%%%%%%%%%%%%%%%%%%%%%%%%%%%%%%%%%%%%%%%%%%%%%%%%%%%%
\begin{document}
\zlabel{iniciodedocumento}
\textbf{}
\newpage

\begin{flushright}
\textbf{\Huge Contenido}
\end{flushright}

\renewcommand*\contentsname{}
{% color de hyperlinks negro para el índice
\hypersetup{linkcolor=black, urlcolor=black,linkbordercolor=black, urlbordercolor=black, pdfborderstyle={/S/U/W 0.5}}
\tableofcontents
}



\newpage
%\pagenumbering{arabic} % empieza la numeración de página

\section{Procedimiento}

Maya 1

\begin{equation*}
10+1kI_1-3.9k(I_1-I_2)-1.8kI_2=0
\end{equation*}
\begin{equation*}
0-6.7kI_1+3.9kI_2=0
\end{equation*}
\begin{equation*}
6.7I_1-3.9I_2=10
\end{equation*}

Maya 2

\begin{equation*}
 -2.2kI_2-2.2kI_2-3.9k(I_2-I_1)=0 
\end{equation*}
\begin{equation*}
 -8.3kI_2+3.9kI_2=0
\end{equation*}

Sistema de Ecuaciones con (1) y (2)

\begin{equation*}
6.7I_1-3.9I_2=10 \\
(1)
\end{equation*}
\begin{equation*}
3.9I_1-8.3I_2=0 \\
(2)
\end{equation*}

Resolviendo:

\begin{equation*}
I_1=2.053 mA
\end{equation*} 
\begin{equation*}
I_2=0.965 mA
\end{equation*}.

Para encontrar los voltajes ocupamos la ley de Ohm

\begin{equation*}
V=I*R
\end{equation*}
\begin{equation*}
VR_1=2.053mA*1K
\end{equation*}
\begin{equation*}
VR_1=2.053 V
\end{equation*}

Para encontrar el voltaje en la resistencia 2 debemos hacer una diferencia entre la corriente 1 y la corriente 2

\begin{equation*}
IR_2=I_1-I_2
\end{equation*}
\begin{equation*}
IR_2=2.053 mA-0.965 mA
\end{equation*}
\begin{equation*}
IR_2=1.088 mA
\end{equation*}
\begin{equation*}
VR_2=1.088 mA*3.9K
\end{equation*}
\begin{equation*}
VR_2=4.24V
\end{equation*}
\begin{equation*}
VR_3=0.965 mA*2.2K
\end{equation*}
\begin{equation*}
VR_3=2.12V
\end{equation*}
\begin{equation*}
VR_4=0.965 mA*2.2K
\end{equation*}
\begin{equation*}
VR_4=2.12V
\end{equation*}
\begin{equation*}
VR_5=2.053mA*1.8k
\end{equation*}
\begin{equation*}
VR_5=3.69V
\end{equation*}

Para poder calcular el error debemos aplicar la siguiente formula:

\begin{equation*}
Error=\frac{v_{teórico}-v_{medido}}{v_{teórico}}100\verb+%+
\end{equation*}

\end{document}